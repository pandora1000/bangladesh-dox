\documentclass{article}
\usepackage{geometry, ulem, microtype, hyperref, dirtree}
\geometry{a4paper, left=20mm, right=20mm, top=10mm}
\title{\textbf{NSDAP Economic Reforms}}
\author{Christ Amlai}
\date{\today}
\begin{document}
\maketitle
\par\rule{\textwidth}{1pt}
\pagenumbering{gobble}
\section*{Context}
Unemployment peaked at 6 million in Germany during 1939, which was 33\% of its population.
The NSDAP devised a set of policies to drastically reduce this figure. One such policy
which can be likened to that of a command economy was for those unable to find work/not searching
for work to be assigned work by the state. The penalty for refusal being either labelled as
``work-shy'', a nudge to encourage more people to enter the labour market. The average
factory worker's wages were on average 10x higher than someone
receiving state handouts. Ultimately, this shifted the supply of
labour outwards. This policy saw unemployment fall to just 302,000 in January 1939.

\section*{Government intervention \& Market Failure}
Defeat in WW1 and The Great Depression had crippled the German economy to a
pint in which German production capacities were reduced 10 percent. Moreover, during WW1
Germany did not see an expansion of capital. Reduced factors of CELL resulted
in a decrease in production and saw Germany working within the PPF.

Market failure can be oberved through the presence of a free-market.
A German citizens ability to consume goods (food, clothing) and services (healthcare)
depended on their income. Consumers with low-income, which composed most
of the German population, could not access this market - thus meet basic needs.
This meant that the market was underproviding these goods. The underprovision
can be seen as a form of Market Failure.

The state corrected market failure through the National Labour Service. The NLS
provided work such as; digging irrigation ditches for farmers, constructing autobahns
and planting new forests. This saw productive efficiency increase, real output increasing
and food prices decreasing. This meant the food market was more accessible to people of
lower income levels. Moreover, the state discouraged nutrional demerit goods such as white bread
and butters for healthier alternatives such as brown bread and potatoes. Ultimately,
discouraging certain foods and planting new forests can be seen as a positive consumption
externality, the spillover to third parties being a healthier lifestyle.

Moreover, the German Labour Front ensured workers could not be sacked on the spot,
working hours incresaed from 6o to 72 increasing production, shifting supply outwards and
protected worker's rights.


\section*{Critiques}
Critiques of the NSDAP's policy's may argue that government-assigned labour/specialisation may see
a decrease in worker productivity as workers may become bored.

However, the Kraft Durch Freude ensured workers
had 3740 hours for leisure - which the state provided. Moreover the kdF involved itself in introducign schemes where
workers could own a car. Workers paid 5 marks a week into an account where they could purchase a Volkswagen Beetlefor 990
marks. A two-week tour of Italy costed only 155 marks. This suggests the negatives of
government-assigned labour/specialisation are ofset by the leisure opportunities.


\newpage

\section*{Summary}
Reasons for scrutiny
\dirtree{%
	.1 Policy summary.
	.2 Women not included in unemployment statistics.
	.2 1935.
    .3 Jewish Germans lost citizenship therefore weren't included in unemployment statistics.
    .3 Conscripted men taken of unemployment figure.
    .3 Fear of criticism by Gestapo.
}

\section*{Sources}
\url{https://www.historylearningsite.co.uk/nazi-germany/the-nazis-and-the-german-economy/}

\url{lade.pdf}
\end{document}
