\documentclass{article}
\usepackage{geometry, ulem, microtype, hyperref}
\geometry{a4paper, left=20mm, right=20mm, top=10mm}
\title{Myanmar}
\author{Christ Amlai}
\date{\today}
\begin{document}
\maketitle
\par\rule{\textwidth}{1pt}
\section*{Myanmar, the \textit{quick rundown}}
Myanmar is a developing country situated in Mainland Southeast Asia.
With a population projected to reach 55 million by the end of 2022
of which according to 2017 data - 24.8\% of the population is
below the national poverty line. Their country has been thrust into
multiple civil wars spanning the past few decades, faced local to national
political insurgences... To list all the problems facing Myanmar and how they
are indeed constraints on development would be exhaustive yet constant links will be made with development indicators and notable historical and current events.

\section*{Development Indicators}
\begin{itemize}
    \item GNI - 253.1 billion PPP dollars (2020)
    \item Literacy rate - 75.55\% DECLINE of 14.39\% from 2000.
    \item Mortality rate - 8.3 per 1000 people
    \item Clean water - 82\% of population
    \item Unemployment - 1.79\%
\end{itemize}
\subsection*{Migration}
Post 2010, 600,000 ethnic Rohingya people have fled to Bangladesh.
Many critics described what followed as an ``unfriendly business environment'',
a massive decrease in the pool of available labour saw an increase in time taken for many social-welfare projects, i.e housing, hospitals, road maintenance.
The mass exodus of Rohingyan people have left much of the agricultural land un-worked. Myanmar is heavily dependent on agricultural land, requiring human capital to work the land. This saw real GDP fall by 15\% in 2022.
\subsection*{Solution through State?}
One controversial solution proposed by many political commentators, is for the state to clamp down on so called, ``clearance operations'' which targeted Rohingyan people, forcing many of them to flee their villages. Such conditions for Rohingyan people to reintigrate into Burmese society and enter the labour force peacefully currently does not exist. Legislature protecting the rights of workers regardless of ethnic-makeup, could be made part of their constitution, guaranteeing Rohingyan people the right to live in safety.
\subsection*{Agriculture}
Seventy percent of Myanmar live in rural townships, their livelihoods
and disposable income are derived from such production. In fact agriculture has always
traditionally been Myanmar's strongest industry, employing half of the labour force in 2019. Moreover, Myanmar was once the biggest exporter of rice in Asia. However, several civil wars, historical and current have led to occupation of local rice fields. This has led to the production volume of rice been stagnant. Less rice can be exported as a result, leading to a decrease in exports.

\subsection*{Agricultural solutions?}
One proposed solution is extended funding of \$1.12 billion dollar loan farmers received from the Myanmar Agricultural Development Bank. Such loans could offset the difficulties posed by the civil wars.

\end{document}
